% Hello!  This document is written to help you learn LaTeX for writing mathematical documents. 
% A LaTeX document is included below.  
% It is intended to get you started with LaTeX, and then to act as sort of a ``quick reference'' 
% for commands after you get comfortable with LaTeX.

% To get started, you will need a TeX distribution and a TeX editor installed on your computer.  
% For a TeX distribution, I recommend TeXLive for Linux, MacTeX for Macs (usually included with TeXShop), and MiKTeX for Windows. 
% As for a TeX editor, TeXMaker is a great editor, which is free, and works on
% Linux, Man, and Windows.  My favorite editor is Kile, which is available for
% Linux.  
% You can find these easily online using Google.  
% Just download and install the TeX distribution (this can be kind of a pain)
% and then install the TeX editor of your choice.
% Next, copy & paste this document into the editor, and compile it by clicking on "LATEX" 
% (your editor may have a different button or menu item).  
% Then click on "DVI Previewer" (or similar).  
% When you are done previewing a document and you are ready to finalize it, you'll want to make a *.pdf file.  
% To make a *.pdf, click on PDF LATEX (or a similar button).  
% Note that this document (the one you are reading right now) is meant to be read 
% in both *.tex form (i.e. uncompiled) and *.pdf form (i.e., compiled).  
% Try compiling it first so that you can see what it looks like, 
% then go back and see how to do those things by looking at this file.  
% Also, notice my hints and tips within this document, 
% which are hidden from the main document using ``%'' symbols to comment them out.  
% Here are a few online resources to get you started:

%   http://www.andy-roberts.net/misc/latex/index.html
%   http://www.andy-roberts.net/misc/

% If you want to use citations and make a bibliography, 
% I strongly recommend looking into the amsrefs package or the BibTeX add-on.  
% You do NOT have to type in bibliography references be hand.  
% Instead, look up the article or book you want on a site like http://www.ams.org/mathscinet/ .  
% Click ``Select Alternate Format'', choose ``BibTeX'', 
% and it will automatically generate an entry for you which you can copy and paste.

% For those of you working on presentations, if you are using programs like OpenOffice Presentation 
% or Microsoft Powerpoint, you will likely find that it can be difficult or impossible to implement 
% the mathematics you want and have things look right.  
% The solution to this is to use Beamer, which is a LaTeX-based presentation format and very 
% standard for giving math talks.  
% If you know LaTeX, Beamer is pretty straight-forward.  A good place to start learning about it is:

%   http://www.math.umbc.edu/~rouben/beamer/

% Happy TeXing!

% The main document follows below.

% --------------------------------------------------------------
% -------------------------Header-------------------------------
% --------------------------------------------------------------
% Every document in LaTeX starts with a ``header'' or ``preamble.''  
% The header defines the way your document looks, and can be a very powerful tool for formatting 
% and custom making commands.  Don't pay much attention to it now, 
% but come back to the header once you feel more comfortable with LaTeX.  
% For now, just copy the full header into your new documents.  
% The end of the header is marked by the ``\begin{document}'' command below.

% Line spacing ----------------------------------------
% This sets up the font size and document type.
\documentclass[11pt]{amsart}

% Document class possibilities:
%   amsart, article, book, beamer, report, letter
% Options:
%   letterpaper, a4paper,11pt,oneside, twoside, draft, twocolumn, landscape

%% For beamer class, see my beamer template.

%% ========== Options to Toggle When Compiling ==============
%\usepackage[notcite]{showkeys} %% Show tags and labels.
%\usepackage{layout}            %% Show variable values controlling page layout.
%\allowdisplaybreaks[1]         %% Allow multiline displays to split.
%\nobibliography     %% Use proper citations, but do not generate bibliography.

%% ========== Select *.tex file encoding and language ==============
%\usepackage[language]{babel} %% Takes care of all language requirements.

%\usepackage[latin1]{inputenc}  %% Use with PuTTY or TeXMaker
% \usepackage[utf8]{inputenc}  %% Use on most OS's, such as Ubuntu.

%% ============== Page Layout ==============
%% Allow extra space at the bottom of pages.
\raggedbottom     

%\usepackage{fullpage}
%% The ``fullpage`` package uses smaller page margins.  
%% It is uses up most of the page and to look nice for homework, 
%% but you should leave it off if you are writing a paper.

%%Control page number placement.  \thepage is the current page number.
% \renewcommand{\headrulewidth}{0pt}
% \lhead{}
% \chead{}
% \rhead{}
% \lfoot{}
% \cfoot{\thepage}
% \rfoot{}

%\usepackage{geometry}  %% Can adjust the margins of individual pages
%% Use it like this:
%% \newgeometry{left=3cm,bottom=0.1cm}
%%     ... Lines that require margins adjusted ...
%% \restoregeometry

%% ============== Page Styles ==============
% \usepackage{fancyhdr}
% \pagestyle{fancy}
% \pagestyle{empty}

%% ============== Math Packages ==============
% Packages define many useful symbols and commands in LaTeX.
% The packages amsmath,amsfonts, and amssymb are essential to most
% math documents.  You can download more packages online, or install
% nearly all of them by installing the full version
% of you TeX distribution.
\usepackage{amsmath}
\usepackage{amsfonts}
\usepackage{amssymb}
\usepackage{amsthm}
%\usepackage{mathtools} % An improvement of amsmath
\usepackage{latexsym}

%% ============ Typesetting add-ons ============
%\usepackage{siunitx} %Support for SI units, \num, \SI, etc.

%% ============== Single-Use Packages ==============
\usepackage{enumerate}
\usepackage{cancel} % Draw diagonal lines trough math to "cancel" things
%\usepackage{empheq} % Make nice braces, etc. around multiline math
\usepackage{cases} % More common, but simpler than empheq.
\usepackage{multicol}

%% ============== Graphics Packages ==============
%\usepackage{graphicx} %% Conflicts with pdflatex.
%\usepackage{graphics} %% Conflicts with eps files.
%\usepackage{epsfig} Allows eps files (?)

%\usepackage{wrapfig} % Make text wrap around figures.

%% Note: For using .eps graphics, use the graphicx package,
%% and in the document use, for example:
%% \begin{figure}
%%  \includegraphics[scale=0.5]{my_picture.eps}
%% \end{figure}

%% Prevent figures from appearing on a page by themselves:
%\renewcommand{\topfraction}{0.85}
%\renewcommand{\textfraction}{0.1}
%\renewcommand{\floatpagefraction}{0.75}

%% Force floats to always appear after their definition: 
%\usepackage{flafter}

%% ============== tikZ and PGF packages ==============
%% tikZ is a way to draw figures entirely with LaTeX code.
%% For more information, see http://www.texample.net/tikz/
% \usepackage{ltxtable,tabularx,tabulary}
% 
% \usepackage{tikz}
% \usepackage{pgf}
% \usepackage{pgfplots} %% Requires pgf 2.0 or later.
% \usetikzlibrary{arrows, automata, backgrounds, calendar, chains,
% matrix, mindmap, patterns, petri, shadows, shapes.geometric,
% shapes.misc, spy, trees}


%% ============== Colors ==============
%% Warning: These are often a source of conflicts during compilation.
% \usepackage{color}
% \newcommand{\blue}[1]{{\color{blue} #1}}
% \newcommand{\red}[1]{{\color{red} #1}}

%% ============== Notes ==============
%% Make notes in your document.
\usepackage{todonotes}
% \listoftodos, \todo[noline]{}, \todo[inline]{}, 
% \todo{}, \missingfigure{}
   
%% ============== Fonts ==============
% \usepackage{bbm}  %% Non-Vanilla: Not include in many LaTeX distros.
\usepackage{mathrsfs}
%\usepackage{fontenc} %T1 font encoding
%\usepackage{inputenc} %UTF-8 support
%\usepackage{babel} %Language specific commands, shortcuts, hyphenation.

\usepackage{verbatim}

%% Microtype improves spacing.  Load after fonts.
%\usepackage{microtype}

%% ============== Theorem Styles ==============
%% These commands define various theorem-like environments.
%% Note: newtheorem* prevents numbering.

\theoremstyle{plain}
\newtheorem{theorem}{Theorem}[section]
\newtheorem{proposition}[theorem]{Proposition}
\newtheorem{lemma}[theorem]{Lemma}
\newtheorem{corollary}[theorem]{Corollary}
\newtheorem*{claim}{Claim}

\theoremstyle{definition}
\newtheorem{definition}[theorem]{Definition}
\newtheorem{example}[theorem]{Example}
\newtheorem{exercise}[theorem]{Exercise}
\newtheorem{axiom}[theorem]{Axiom}

\theoremstyle{remark}
\newtheorem{remark}[theorem]{Remark}

%% ============== References ==============
\numberwithin{equation}{section} %% Equation numbering control.
\numberwithin{figure}{section}   %% Figure numbering control.

%% Natbib can sort and compress your \ref numbering.
\usepackage[square,comma,numbers,sort&compress]{natbib}
%\usepackage[colorlinks=true, pdfborder={0 0 0}]{hyperref}
%\hypersetup{urlcolor=blue, citecolor=red}
\usepackage{url}

%% Reference things as 'fig. 1', 'Lemma 7', etc.
%% Note: some conflict with section labeling
% \usepackage{cleveref}

%% Create references like 'on the following page', 'on page 23'
% \usepackage{varioref} 

% usepackage[refpages]{gloss} %% Glossary

%%%%%%%%%%%%%%%%%%%%% MACROS %%%%%%%%%%%%%%%%%%%%%
%% Sometimes it can be very tedious to retype long
%% commands.  Use ``new command'' to avoid this.
%% For example, here you only need to type $\nR$
%% in your document rather than $\mathbb R$ to
%% get the symbol for the real numbers.

% ============================== Vectors ==============================
\newcommand{\vect}[1]{\mathbf{#1}}
\newcommand{\bi}{\vect{i}}
\newcommand{\bj}{\vect{j}}
\newcommand{\bk}{\vect{k}}

\newcommand{\bu}{\vect{u}}
\newcommand{\bv}{\vect{v}}
\newcommand{\bw}{\vect{w}}
\newcommand{\boldm}{\vect{m}}
\newcommand{\bx}{\vect{x}}
\newcommand{\by}{\vect{y}}
\newcommand{\bz}{\vect{z}}

\newcommand{\be}{\vect{e}}

% ==================== Fields ==================
\newcommand{\field}[1]{\mathbb{#1}}
\newcommand{\nC}{\field{C}}
\newcommand{\nF}{\field{F}}
\newcommand{\nK}{\field{K}}
\newcommand{\nN}{\field{N}}
\newcommand{\nQ}{\field{Q}}
\newcommand{\nR}{\field{R}}
\newcommand{\nT}{\field{T}}
\newcommand{\nZ}{\field{Z}}

% ======================== Script Symbols  ========================
\newcommand{\sL}{\mathscr L}
\newcommand{\sH}{\mathscr H}

% ====================== Caligraphic Symbols ======================
%% Make your caligraphy fonts even fancier with this:
% \usepackage{eucal}
\newcommand{\cA}{\mathcal A}
\newcommand{\cB}{\mathcal B}
\newcommand{\cC}{\mathcal C}
\newcommand{\cD}{\mathcal D}

\newcommand{\cL}{\mathcal L}

% ======================== Fraktur Symbols  ========================
% Note: You must use the mathrsfs package for these symbols.

%\newcommand{\fM}{\mathfrak M}

% ========================== Bold Symbols ==========================
\newcommand{\bvphi}{\boldsymbol{\vphi}}
\newcommand{\bPhi}{\boldsymbol{\Phi}}

% ======================== Misc. Symbols ========================
\newcommand{\QED}{\hfill$\blacksquare$}
% \QED is not recommended.  Instead, use \begin{proof}...\end{proof}.

% For a fancy lowercase l, use \ell

% You can even make your own symbols (advanced):
\newcommand{\dhr}{\m\athrel{\lhook\joinrel\relbar\kern-.8ex\joinrel\lhook\joinrel\rightarrow}}

% ========================== Operations ==========================
%% These can be very useful.  Note how they are defined.
%% The first bracket {} names the new command.  The second []
%% is the number of arguments, and the third {} is the full
%% length expression.  For example, rather than writing
%% out the full notation for the partial derivative, of
%% f with respect to x, the command below allows you
%% to just write $\pd{f}{x}$.  This can be a big time saver
%% and can make your .txt file easier for you to read.
\newcommand{\cnj}[1]{\overline{#1}}
\newcommand{\pd}[2]{\frac{\partial #1}{\partial #2}}
\newcommand{\npd}[3]{\frac{\partial^#3 #1}{\partial #2^#3}} %\npd{f}{x}{2}
\newcommand{\abs}[1]{\left\lvert#1\right\rvert}
\newcommand{\norm}[1]{\left\lVert#1\right\rVert}
\newcommand{\set}[1]{\left\{#1\right\}}
\newcommand{\ip}[2]{\left<#1,#2\right>}
\newcommand{\pnt}[1]{\left(#1\right)}
\newcommand{\pair}[2]{\left(#1,#2\right)} 

% ============ Special Macros For This Paper ==================

%% Put your macros here.

%% ============== Counters ==============
\newcounter{my_counter}    %% Define a new counter.
\setcounter{my_counter}{1} %% Set is equal to 1.

 %% ============== Article Information ==============
\title[Short Version of Tile]{Full Title Goes Here}

\date{\today}
% \date{3 July 2013}

\keywords{keyword1, keyword2,...}
\thanks{MSC 2010 Classification: }
%% Note: Find MSC classification numbers at
%% http://www.ams.org/mathscinet/freeTools.html

% ======================  Author Information ======================
\author{First Author}
\email[First Author]{student1@MyEmail.edu}
%
\author{Second Author}
\email[Second Author]{student2@MyEmail.edu}
%

%%%%%%%%%%%%%%%%%%%%%%%%%%%%%%%%%%%%%%%%%%%%%%%%%%%%%%%%%%%%%%
\begin{document}
%%%%%%%%%%%%%%%%%%%%%%%%%%%%%%%%%%%%%%%%%%%%%%%%%%%%%%%%%%%%%%
\thispagestyle{empty} %% Gets rid of page number on first page.

\begin{abstract}
The abstract goes here. 
\end{abstract}
   
\maketitle
   
\section{Introduction}\label{sec:intro}

Here is some stuff in the introduction.  It will lay out all the other things.

\section{Why Math is Awesome}\label{sec:awesome}

In this section, we will explain why math is awesome.  First, how about a system of equations in display-style:
\begin{subequations}\label{Navier_Stokes}
\begin{align}
\label{momentum_equation}
 \partial_t u +(u\cdot\nabla)u+\nabla p &= \nu\triangle u +f,
\\
\label{MHD_div}
 \nabla\cdot u &= 0,
\\
\notag
 u(0) &= u_0.
\end{align}
\end{subequations}
% Notice the & symbols separating the columns.
% Also \\ means "end the current line."
% You don't need a ''\\" on the last line.  
% In fact, you shouldn't put one there
% since it will make an additional blank row.

% To get rid of the numbering, use a star like this:  
% \begin{align*} ... \end{align*}
% Or use \notag to get rid of the number on a specific line.

% Note: NEVER leave a blank line in your align environment. 
% An error will happen when you try to compile.

% There is an old way of making multi-line equations, called ``eqnarray''
% It is now considered obsolete, and you shouldn't use it.

\subsection{Why math is fun}\label{subsec:fun}

Here is a subsection.  We can reference the previous system of equations \eqref{Navier_Stokes}, or specific parts of it \eqref{momentum_equation} or \eqref{MHD_div}.  

This is a citation of a paper \cite{Beale_Kato_Majda_1984} and a book \cite{Chandrasekhar_1961}, which will show up in the bibliography.  Remember that an easy way to get formatted references is to go to \url{http://www.ams.org/mathscinet} from a university computer, look up an article, click ``Select Alternative Format'', and select ``BibTeX.''  Then just copy the reference into your $*$.bib file.



\subsection{Why math is cool}\label{subsec:cool}

Here is another subsection.

\section{Some Additional Tips}\label{sec:latex}

In Section \ref{sec:awesome}, you saw how to make multi-line equations.  Single-line equations can be done the same way, but there is also simpler way:
\[
\int_{\Omega}df = \oint_{\partial\Omega}f(x) \, dx
\]

% Note the ``\,'' symbol.  This makes a small space so the ``f(x)dx'' looks better.
% LaTeX doesn't pay attention to spacing in the tex document, so 
% $y = x$ is the same as $y         =        x$.

% Note also that the old double dollar sign notation still works, but is outdated.  
% That is, the above line could have been written as:
% $$\int_{\Omega}df=\oint_{\partial\Omega}f$$
% In fact, it is preferable to use \( ... \) notation instead of $ ... $, but 
% the single dollar sign is used so common, that I will use it here as well.

% To get rid of the numbering, use a star like this:  
% \begin{align*} ... \end{align*}
% Or use \notag to get rid of the number on a specific line.

% Note: NEVER leave a vertical space in your equation array. An error will happen when you try to compile

\begin{enumerate}
  \item You can easily make lists using \emph{enumerate}.
  \item The numbering is done automatically.

    \begin{enumerate}
      \item You can even make lists within lists.
      \item I heard you like lists, so I put a list in your list.
    \end{enumerate}

\end{enumerate}

Numerical lists are handy, but sometimes, you wanted a bulleted list:

\begin{itemize}
  \item[$\spadesuit$] Or you can make your own symbols using \emph{itemize}.
  \item[$\bigotimes$] Even unusual symbols.
  \item The default symbol is a standard bullet.
\end{itemize}

There are many way to make matrices.  Notice the use of the left and right operators.  They will stretch to any size you like.
\[
\left[
\begin{array}{rl|c}%r=right, l=left |=make a line, c=center
3 & 1 & 4 \\
1 & 5 & 9 \\
\hline
2 & 6 & \text{See!  Easy.}
\end{array}
\right]
\]
% Notice the \left[  and \right] notation.  
% You can use this in other ways, such as \left( and \right), or even \left\{ and \right. 

\noindent
You can also make ``piecewise'' defined functions:
%Notice how you have to use different symbols for the left and right quotation marks.  
%If I had written "piecewise" above instead of ``piecewise'' the document looks wrong.  
%Some editors correct for this automatically.  
%The ` symbol is near the upper-left part of your keyboard, while the ' symbol 
%is the same key as " except that you don't press the shift button.
\[
\Phi(x):=
\left\{
\begin{array}{ll}
\sum_{n=0}^\infty\frac{x^{-2n}}{(2n+1)!}  & \text{if }x\in\mathbb{R}\setminus\{0\},\\
0 & \text{otherwise.}
\end{array}
\right. 
%The "." is an empty character in this situation, 
%but it is needed so that the "\left" and "\right" commands match.
\]
%Note that you can write the symbol for the real numbers as $\mathbb{R}$.  
%However, because of the ``new command'' defined in the header, you can also write $\nR$ to get it.

%Note the symbol {rc|r} means "first column aligned right"
%"second column aligned center"
%Then a vertical line
%"third column aligned center"

\noindent There are a few ways to cancel things: 
$a\neq b$, $a\not\in A$, $\cancel{(x^2+1)}y=\cancel{(x^2+1)}z$ (you need the \emph{cancel} package for that last one).    
You can write fractions like this: $\frac{z+i}{z-i}$.  
If you want to make them look like display text, try $\dfrac{z+i}{z-i}$.  
More generally, try \emph{display style}: 
$\displaystyle{\lim_{n\rightarrow\infty}a_n}\neq b$. 
Surely \LaTeX would have been loved by great mathematicians, such as L'H\^ospital, H\"older, or Poincar\'e! (Note how \LaTeX can easily make accents on the names.)

\LaTeX has many built-in symbols.  Most of them can be found in your TeX editor's symbol list.  For standard calculus functions, \LaTeX has built-in symbols, which you should use.  For example, 
\[
 \sin(x) \text{ and } \log(x)
 \qquad\text{ look better than }\qquad
 sin(x) \text{ and } log(x).
\]

\section{Online Resources}

\begin{itemize}
\item Google is your friend.  Remember to type ``latex'' into Google followed by your query.
\item For \LaTeX presentations (like Powerpoint, but better-looking), use the ``Beamer'' package:\\
\url{http://en.wikibooks.org/wiki/LaTeX/Presentations}\\
Or, for a beamer quickstart:\\
\url{http://www.math.umbc.edu/~rouben/beamer/}
\item For a guide to inserting pictures/tables, etc., look here:\\ 
\url{http://www.andy-roberts.net/writing/latex}
\item To make 2D or 3D graphics/plots/etc. directly in \LaTeX, have a look at:\\
\url{http://www.texample.net/tikz/}

\end{itemize}


\begin{equation*}
|
\big[\Big<\bigg\{\Bigg(
\underline{\hspace{1.2cm}}
/ \text{\Large{Happy TeXing!}} \backslash
\underline{\hspace{1.2cm}}
\Bigg)\bigg\}\Big.\big]
|
\end{equation*}


%%%%%%%%%%%%%%%%%%%%%%%%%%%%%%%%%%%%%%%%%%%%%%%%%%%%%%%%%%%%%%
%% You must download ``Larios_Bibliography.bib'' and put it in 
%% the same folder as this document to be able to compile.
\begin{scriptsize}
\bibliographystyle{amsplain}%amsalpha%amsplain%plain
\bibliography{Larios_Bibliography}
\end{scriptsize}

\end{document}
%%%%%%%%%%%%%%%%%%%%%%%%%%%%%%%%%%%%%%%%%%%%%%%%%%%%%%%%%%%%%%