\documentstyle[11pt]{amsart}


\topmargin=-.5in
\textheight=9in
\oddsidemargin=0.2in
\textwidth=6.0in
\pagestyle{empty}
\begin{document}
\begin{center}
{\bf MATH 609-600 \\
Homework \# 4 \\
Polynomial interpolation of functions of one variable }
\end{center}

\def\R{{\mathcal R}}

\bigskip
\noindent
Solve any set of problems for 100 points. 
The homework should be presented at the beginning of the class.
There penalty for delay of the homework is 5 pts per day.

\bigskip
\noindent
\begin{enumerate} 

\bigskip
\item (20 pts) Find the Lagrange and backward Newton divided difference
interpolating polynomials for the data $(0,1),~(0.5, 2),~(1, 3),~(1.5, 4)$.


\bigskip
\item (20 pts) Estimate the interpolation error of $\cos x$
in the interval $(0, 0.4)$ by a polynomial of degree 2 using the 
interpolation nodes $x_0=0,~x_1=0.2,~x_2=0.4$.

\bigskip
\item (20 pts) Prove the identities $\sum_{k=0}^{n}(x-x_k)^m l_{n,k} (x) =0$ for $m=1,...,n$. The basic polynomials $ l_{n,k} (x)$ were defined in class as
$$
l_{n,k} (x) = \frac{(x-x_0) \dots (x-x_{k-1})(x-x_{k+1}) \dots (x-x_n)}{(x_k-x_0) \dots (x_k-x_{k-1})(x_k-x_{k+1}) \dots (x_k-x_n)}.
$$ 

\item  Let $f(x)=x^n $ and $f[x_0,x_1,...,x_n]$ be the divided
difference of order $n$ using the points $x_0 < x_1 <... < x_n$. Prove that:
\begin{enumerate}
\item (10 pts) $f[x_0,x_1,...,x_n]=1$; 
\item (10 pts) $f[x_0,x_1,...,x_{n-1}]=x_0 +x_1 + ... + x_{n-1}$.
\end{enumerate}


\bigskip
\item (10 pts) If $f[x_0,x_1, \dots, x_n]$ denotes the divided difference
of order $n$ prove the {\bf Leibnitz formula}
$$(fg)[x_0,x_1, \dots, x_n]= 
\sum_{k=0}^n f[x_0,x_1, \dots, x_k]g[x_k,x_{k+1}, \dots, x_n].$$



\bigskip
\item (20 pts) Let  $p(x)$ be the Hermite intepolating polynomial based on
$n+1$ distinct points $x_0 < \dots < x_n$ in $[a,b]$. Assume that the data 
$f_i=f(x_i)$ and $f'_i=f'(x_i)$ is generated by a function 
$f(x) \in C^{(2n+2)}([a,b])$. Prove that for each $x \in [a,b]$ there is a point $\xi_x$ such that
$$
f(x)= p(x)+ \frac{1}{(2n+2)!}f^{(2n+2)}(\xi_x) (x-x_0)^2 \dots (x-x_n)^2.
$$

\bigskip
\item (20 pts) Prove that there is unique polynomial of degree at most $n+2$ that interpolates the data:
$f(x_0), f'(x_0), f(x_1), f(x_2), \dots, f(x_{n-1}), f(x_n), f'(x_n)$ at the $n$ distinct points
$x_0< \dots < x_n$.

\bigskip
\item In your free time and for your amusement(!!!):
%\begin{enumerate}
%\item   (10 pts) $\sum_{k=0}^{n}(x-x_k)^m L_{n,k} (x) =0$ for $m=1,...,n$.
%\bigskip
 Show that if 
$$\omega (x)=(x-x_0)(x-x_1)...(x-x_n)$$ 
then:
\begin{enumerate}
\item  $\sum_{k=0}^{n}(x-x_k)^{n+1} l_{n,k} (x) = (-1)^n \omega (x)$ 
\item  $\sum_{k=0}^{n}(x-x_k)^{n+2} l_{n,k} (x) = 
(-1)^n \omega (x) \sum_{k=0}^{n} (x-x_k)$ 
\item  $ \sum_{k=0}^{n} l_{n,k} (0) x_k^{n+1}=(-1)^n x_0.x_1...x_n$.
\end{enumerate}

\end{enumerate}


\end{document}


\begin{comment}
\bigskip
\item (20 pts) Let 
$$L_{n,k}(x)= \displaystyle
\frac{(x-x_0)...(x-x_{k-1})(x-x_{k+1})...(x-x_n)}
{(x_k-x_0)...(x_k-x_{k-1})(x_k-x_{k+1})...(x_k-x_n)}. 
$$ 
Show that for any $x$ the following relations are valid:
\begin{enumerate}
\item $\sum_{k=0}^{n} L_{n,k} (x) = 1$;
\item $\sum_{k=0}^{n}x_k^m  L_{n,k} (x) = x^m$ for $m=1,...,n$.
\end{enumerate}

\end{comment}
