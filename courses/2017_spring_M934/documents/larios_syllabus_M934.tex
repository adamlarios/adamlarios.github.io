\documentclass[margin]{res}
\setlength{\textheight}{9.5in}

% --------------------- Compiling Options -----------------------
%\usepackage[notcite]{showkeys} %Show tags and labels.
%\numberwithin{equation}{section} % Equation numbering control.
%\nobibliography %Use proper citations, but do not generate bibliography.
%\allowdisplaybreaks[1] %Allow multiline displays to split.
%\raggedbottom

% ------------------------ Basic Packages ------------------------
\usepackage{amsmath,amsfonts,amssymb,amsthm,latexsym}
\usepackage{enumerate}
\usepackage{cancel}
%\usepackage{ulem}
\usepackage{empheq}
\usepackage{verbatim}
\usepackage{datetime}
\shortdate
%\longdate,\shortdate,\usdate,\ddmmyyyydate,\dmyyyydate,\ddmmyydate,\dmyydate,
%\textdate,\mmddyyyydate,\mdyyyydate,\mmddyydate,\mdyydate,\yyyymmdddate

% \usepackage[textwidth=5cm,textheight=9.5in]{geometry}
% \setlength{\textheight}{9.5in}

% --------------------- Advanced Packages ----------------------
\usepackage{cases}
%\usepackage[square,comma,numbers,sort&compress]{natbib}
%\usepackage{empheq}

% \usepackage[colorlinks=true]{hyperref}
% \hypersetup{urlcolor=blue, citecolor=red}

\usepackage{textcomp}
\usepackage{url}

% ---------------------- Graphics Packages -----------------------
%\usepackage{graphicx}
%\usepackage{graphics}
%\usepackage{fancyhdr}
\usepackage{wrapfig}

%Note: For using .eps graphics, use the graphicx package above,
% and in the document use, for example:
% \begin{figure}
%  \includegaphics[scale=0.5]{my_picture.eps}
% \end{figure}

% -------------------------------- Notes ------------------------------
 % Use in document as \mnote{My note here.}
% \newcounter{mnote}
%  \setcounter{mnote}{0}
%  \newcommand{\mnote}[1]{\addtocounter{mnote}{1}
%    \ensuremath{{}^{\bullet\arabic{mnote}}}
%    \marginpar{\footnotesize\em\color{red}\ensuremath{\bullet\arabic{mnote}}#1}}


% -------------------------------- Fonts ------------------------------
%\usepackage{bbm}
%\usepackage{mathrsfs}

% ------------------------  Theorem Styles ------------------------
% Note: newtheorem* prevents numbering.

\theoremstyle{plain}
\newtheorem{theorem}{Theorem}[section]
\newtheorem{proposition}[theorem]{Proposition}
\newtheorem{lemma}[theorem]{Lemma}
\newtheorem{corollary}[theorem]{Corollary}
\newtheorem*{claim}{Claim}

\theoremstyle{definition}
\newtheorem{definition}[theorem]{Definition}
\newtheorem{example}[theorem]{Example}
\newtheorem{exercise}[theorem]{Exercise}
\newtheorem{axiom}[theorem]{Axiom}

\theoremstyle{remark}
\newtheorem{remark}[theorem]{Remark}

% ------------------------------ Vectors ------------------------------
\newcommand{\vect}[1]{\mathbf{#1}}
\newcommand{\vi}{\vect{i}}
\newcommand{\vj}{\vect{j}}
\newcommand{\vk}{\vect{k}}

\newcommand{\bu}{\vect{u}}
\newcommand{\bv}{\vect{v}}
\newcommand{\bw}{\vect{w}}

\newcommand{\bx}{\vect{x}}
\newcommand{\by}{\vect{y}}
\newcommand{\bz}{\vect{z}}

% -------------------- Fields -------------------
\newcommand{\field}[1]{\mathbb{#1}}
\newcommand{\nN}{\field{N}}
\newcommand{\nZ}{\field{Z}}
\newcommand{\nQ}{\field{Q}}
\newcommand{\nR}{\field{R}}
\newcommand{\nC}{\field{C}}
\newcommand{\nF}{\field{F}}
\newcommand{\nK}{\field{K}}
\newcommand{\nT}{\field{T}}

% ------------------------- Script Symbols  -------------------------
\newcommand{\sL}{\mathscr L}
\newcommand{\sH}{\mathscr H}

% ---------------------- Caligraphic Symbols ----------------------
\newcommand{\cA}{\mathcal A}
\newcommand{\cB}{\mathcal B}
\newcommand{\cC}{\mathcal C}

\newcommand{\cN}{\mathcal N}

% ------------------------- Fraktur Symbols  -------------------------
% Note: Use mathrsfs package in Fonts section above.

%\newcommand{\fM}{\mathfrak M}

% -------------------------- Bold Symbols --------------------------
\newcommand{\bvphi}{\boldsymbol{\vphi}}
\newcommand{\bPhi}{\boldsymbol{\Phi}}

% ------------------------ Misc. Symbols ------------------------
\newcommand{\vphi}{\varphi}
\newcommand{\maps}{\rightarrow}
\newcommand{\Maps}{\longrightarrow}
\newcommand{\sand}{\quad\text{and}\quad}
\newcommand{\QED}{\hfill$\blacksquare$}
%\newcommand{\dhr}{\mathrel{\lhook\joinrel\relbar\kern-.8ex\joinrel\lhook\joinrel\rightarrow}}

% -------------------------- Operations --------------------------
\newcommand{\cnj}[1]{\overline{#1}}
\newcommand{\pd}[2]{\frac{\partial #1}{\partial #2}}
\newcommand{\fd}[2]{\frac{d #1}{d #2}}
\newcommand{\npd}[3]{\frac{\partial^#3 #1}{\partial #2^#3}}
\newcommand{\abs}[1]{\left\lvert#1\right\rvert}
\newcommand{\norm}[1]{\left\lVert#1\right\rVert}
%\newcommand{\LiL2}[1]{\norm{#1}_{L^\infty((0,T);L^2(\Omega))}}
%\newcommand{\norm}[2]{\left\lVert#1\right\rVert_{#2}}
%\newcommand{\L2}[1]{\norm{#1}{L^2(\Omega)}}
%\newcommand\norm[1]{\left\vert\mkern-1.7mu\left\vert#1\right\vert\mkern-1.7mu\right\vert}
%\newcommand\bnorm[1]{\bigl\vert\mkern-2mu\bigl\vert#1\bigr\vert\mkern-2mu\bigr\vert}
\newcommand{\set}[1]{\left\{#1\right\}}
\newcommand{\ip}[2]{\left<#1,#2\right>}
\newcommand{\pnt}[1]{\left(#1\right)}
\newcommand{\pair}[2]{\left(#1,#2\right)}
%----------------------------------------------------

 \newcounter{week}
 \setcounter{week}{1}
 \newcommand{\week}{\arabic{week}\addtocounter{week}{1}}

 \newcounter{dayNum}
  \setcounter{dayNum}{1}
  \newcommand{\dayNum}{\arabic{dayNum}\addtocounter{dayNum}{1}}
 
\begin{document}
\begin{center}
   \textbf{MATH 934:} 
   \textbf{Topics in Differential Equations}\\
   \textbf{UNL, Spring 2017}, Section: 001%, CRN: 
   %-----------------
    \\
   \textbf{Lecture:} M, W, F, 8:30 am-9:20 am, Avery Hall 351\\
   % 08/25-12/15
   % Credits: 3 units
   % Student Capacity: 30
   % Class email: 
   % Course ID: 042718
   %-----------------
\end{center}
%
\vspace{-0.2in}
%
\begin{resume}
\section{Instructor:} Dr. Adam Larios 
\hfill \textbf{Email:} 
\url{alarios@unl.edu}
\\
%\begin{tabular}{l c l}
\textbf{Office:} Avery Hall 305
\hfill
%&& 
\textbf{Math Dept. Phone:} (402) 472-7250
\\
\textbf{Office Hours:} M,W,F, 2:30 pm - 3:20 pm, or by appointment
\hfill
\\
\textbf{Web:} \url{www.math.unl.educourses/2017_spring_M934/content.html}
%&&

%\end{tabular} 

% \section{Teaching Assistant:} 
% \hfill \textbf{Email:} 
% \url{}
% \\
% %\begin{tabular}{l c l}
% \textbf{Office:} 
% \hfill
%&& 

 \section{Prerequisites:} MATH 314/814 (Linear Algebra), and MATH 325 (Elementary Analysis), or equivalents.  You are also expected to know differentiation and integration techniques from calculus, as well as the material from multivariable calculus.   You are also expected to be able to understand mathematical proofs.  This course will require basic computer skills.  We will learn programming, but prior knowledge of programming is \textit{not} a prerequisite.
 
 
%  A grade of P or C or better in MATH 208/208H (Calc III), or approved equivalent. You are expected to know differentiation and integration techniques and to be familiar with vector fields and parameterized curves.
 
%  MATH 251 or equivalent with a grade of C or better, basic knowledge of computers and the ability to handle programming or computer algebra systems (e.g.,  Sage, Mathematica, Maple, or something similar; or Matlab, C/C++, Python, Java, Fortran, etc.)  We will be using Matlab in this course.

% ==================================================
\section{Textbook:}  \textit{Partial Differential Equations with Numerical Methods}.  Stig Larsson, Vidar Thomee.  Springer, 2008. ISBN: 978-3540887058.

% Optional:
% Numerical Optimization (Springer Series in Operations Research and Financial Engineering) 2nd Edition
% by Jorge Nocedal (Author), Stephen Wright (Author)
%  Springer Series in Operations Research and Financial Engineering

% \textit{Elementary Differential Equations: Custom TAMU Edition}. W. Boyce, R. DiPrima.   John Wiley and Sons, Inc, 2011. ISBN: 9781118133712.  \textbf{Note: this is a special edition made for Texas A\&M. It is a shortened version of the $9^{\text{th}}$ edition.}
% 
% \textit{Differential Equations with Matlab, 2nd Edition}. Wiley, B. Hunt, R. Lipsman, J. Osborn, J. Rosenberg.  John Wiley and Sons, Inc, 2005. ISBN:  9780471718123.  \\\textbf{(Optional)}
% ==================================================

% % ==================================================
% \section{ACE Outcome 3:}``Use mathematical, computational, statistical, or formal reasoning (including reasoning based on principles of logic) to solve problems, draw inferences, and determine reasonableness.'' Your instructor will provide examples, you will discuss them in class, and you will practice with numerous homework problems. The exams will test how well you've mastered the material.  The final exam will be the primary means of assessing your achievement of ACE Outcome 3.
% % ==================================================

% ==================================================
\section{Contacting me:}
% The best way to contact with me is by email, \url{alarios@unl.edu}.  Please put [MATH 433] somewhere in the title and make sure to include your whole name in your email. Polite, courteous emails are appreciated; see my website for tips on email etiquette.  My office is in Avery Hall, room 305, and my office hours are M,W,F, 12:30 pm - 1:20 pm. Drop-ins are welcome during these times.  If you want to meet me at a different time, please email me in advance, and we will try to schedule a time to meet.

\textbf{NOTE:} Because of privacy rights, \textbf{I cannot discuss grades over email or telephone. Please do not email me asking about your grade.  I will not be able to give you any information.}  Of course, I am happy to discuss grades in my office.
% ==================================================

% ==================================================
\section{Description:} We will solve differential equations using numerical methods, and this does not mean that we will leave mathematical rigor or beauty behind!  The subject of numerical PDEs is full of clever ideas, elegant structures, dazzling schemes, and subtle concepts.  The world of PDEs is so vast, that you can spend several lifetimes studying just one PDE, and yet there are thousands of PDEs out there.  They are used to model phenomena such as weather, turbulence, blood flow, cancer growth, traffic, financial markets, ecology, acoustics, electricity, magnetism, star formation, and the bending of spacetime itself.  Solving them not only unlocks new areas of science, but often leads to pretty pictures that can amaze people in your poster sessions and astonish audiences at conferences.

To set the stage, we will begin with numerical solutions of ODEs (ordinary differential equations).  We will quickly move on to numerical solutions of PDEs (partial differential equations).  We will learn spectral/Fourier methods, finite element methods, and, time permitting, several other methods.  We will learn to program in Matlab, and we will also also a finite element library called FEniCS, which uses Python.

No programming background is necessary, and it is not necessary for you to have taken a course in PDEs.  An undergraduate ODE course, such as Math 221, advanced knowledge of linear algebra, such as Math 415/815 (or anything beyond Math 314), and some analysis, such as Math 825/826 (could be taken concurrently), should be sufficient.


% ==================================================

% % ==================================================
%  \section{Motivation:} Nonlinear optimization is a generalization of the material in the multivariable calculus course dealing
% with finding and analyzing critical points, solving global extremum problems, and constrained
% optimization using the Lagrange multiplier rule. The machinery of linear algebra makes it easier to
% state nonlinear optimization problems and discuss the mathematical theory. Most nonlinear problems
% are too complicated to solve by hand, so numerical methods for optimization are an important
% component of any study of the subject. Nonlinear optimization problems can be broadly classified as
% unconstrained and constrained optimization, with similar theory but very different methods.  We will explore tools that have been developed to handle this beautiful and very useful subject.
% 
% % ==================================================

% ==================================================
\section{Projects:} 
% ==================================================
The majority of the course grade will be based on projects.  These will be mostly coding projects, but they may also involve some mathematical analysis.  As codes often either work or do not work, it can be difficult to give a scaled grade.  Therefore, often codes will be checked off once they are working.  

% ==================================================
\section{Exams:}
% ==================================================
There will be no exams or quizzes

% ==================================================
\section{Reading:} 
% ==================================================
Please read the book frequently.  It will help.

% ==================================================
\section{Homework:}
% ==================================================
Overall, we will aim to keep the course load light; however, to learn this stuff, there will be times where you must try things on your own.  Therefore, there will be a few mild homework assignments, but I will do my best to avoid problem sets which burn huge amounts of time, or keep you up all night.  Think of them as ``lunch-time exercises.''  Hopefully, you will be able to take these problems with you, and work on them casually when you have time over a few days.  There is no need for a formal write-up; just discuss the problem with me at some point, and I will check it off.


% ==================================================
\section{Grading:} 
% ==================================================
Points will be associated with the course work.  If your total number of points is within one standard deviation of the course mean, you will receive a grade in the ``A-/A/A+'' range.  Grades of A+ will be reserved for exemplary course work, although it would certainly be possible for all student to receive a grade of A+.  If the total number of points is between 1 and 2 standard deviations below the mean, will receive a grade in the ``B-/B/B+'' range, and so on for each interval of unit standard deviation.  This grading system assumes that the majority of student complete the majority of the projects.  Adjustments may be made in the the unlikely event that this does not occur.

% ==================================================
\section{Attendance \& Preparation:} 
There is no textbook that exactly fits the course goals; hence, the lecture notes are the primary record of the course. Regular attendance and attention is therefore critical. It will be helpful for you to browse through the material before it is presented in class.

Daily attendance for class lectures is expected and is extremely important. 
While attendance is not recorded, missing even one class will put you behind.    You are responsible for all material and announcements in class regardless of whether or not you attended.  You are also responsible for making arrangements with another classmate to find out what you missed. 

If you know ahead of time that you will be gone. Please let me know as soon as possible in advance. Reasonable accommodations will be made for university-excused absences.
% ==================================================

% % ==================================================
% \section{Scheduling:} A tentative schedule of assignments and exams is included in this syllabus. These details are presented as a guide. Your instructor may change the dates for each assignment and/or exam, modify the exercise list, and/or add assignments. It is your responsibility to keep track of the course details and schedule for your section.
% % ==================================================

% ==================================================
\section{Computing:}
We will be writing programs to implement numerical methods. I will prepare directions for writing in
Matlab. You may use your own computing equipment, and you may also use the computers in the Math Department computer lab in Avery 9, or in labs around campus.  Matlab is free to download for UNL students, and can be accessed here:

\texttt{\textbf{http://procurement.unl.edu/matlab-licenses}}

At some point, I will also ask you to install a software library called FEniCS.  Both Matlab and FEniCS can take up several GB, so please plan to have space available (say, 
% ==================================================

% ==================================================
\section{Collaboration:} Collaboration is encouraged in this course. However, copying someone else's work and submitting it as your own is unacceptable.
% ==================================================

% % ==================================================
% \section{Projects:}
% Projects will be assigned as announced in class. For some projects, you may be put into groups.  In these cases, you will be asked to collaborate on and turn in a single assignment, which will receive a single grade.  Each member of the group is responsible for 100\% of each assignment, so if one or more  group members do not fully contribute for any reason, it is still the responsibility of the other group members to turn in a completed assignment by the due date.  It is your responsibility to contact your group members, and to contact me well in advance if problems arise.  
% % ==================================================

% % ==================================================
% \section{Quizzes:}
% There will be weekly (or almost weekly) quizzes administered in class.  \textbf{No make-up quizzes will be given; however, the lowest two quiz scores will be dropped.} You are required to bring and possibly present your N-Card or a government issued ID card when taking exams.
% % ==================================================

% % ==================================================
% \section{Electronics: } You are not allowed to have on your person during exams or quizzes any device that can access the internet or communicate in any way.  Cell phones, Apple watches, etc. should be put away in backpacks/purses.  Calculators, laptops, tablets, cell phones, and other non-medical electronic devices are not permitted during exams unless otherwise stated.  During class, cell phones should be set on vibrate or off.  If you need to take a call, send a text message, etc., please quietly leave the classroom to do so, so that you do not distract other students.  You are welcome to return to class quietly when you are finished.  If you wish to take notes using an electronic device, you must first demonstrate to me that you can type or write fast enough to do so properly, and that you can do it without distracting others, before the privilege to use such devices may be granted.  If you are found to be abusing this privilege, you risk forfeiting it.
% % ==================================================

% \pagebreak
% 
% % ==================================================
% \section{Grading:}
% Your course grade will be based on a weighted average computed as follows.
% \begin{center}
% \begin{tabular}{| l  r  |}\hline
% Homework:   & 30\%                 \\
% Midterms:   & $2\times20\% = 40\%$ \\
% Final Exam: & 30\%                 \\
% Total:      & 100\%                \\ \hline
% \end{tabular}
% \end{center}
% % ==================================================

% ==================================================
% \section{Grading:}
% Your minimal course grade will be computed as follows. All work in the course will be graded according to the following scale:
% 
% A: 90,
% A-: 87,
% B+: 84,
% B: 80,
% B-: 77,
% C+: 74,
% C: 70,
% C-: 67,
% D+: 64,
% D: 60,
% D-: 57


% If deemed necessary, minor adjustments to this scale will be made in favor of the students (commonly known as ``applying a curve'').  A grade of ``A+'' may be assigned in the case truly exceptional work.
% ==================================================

% % ==================================================
% \section{Project:} There will be a Matlab project.  
% Details about the specifics of the project will be provided
% when the project is assigned.  The project is optional, 
% but highly recommended.  The grade of your project can
% be used to replace the lowest 15\% of one of your homework, 
% quiz, or mid-term exam grade categories, with each exam counted separately.
% This will be done automatically (there is no need to request it), 
% and it will only be done if it benefits your grade.  
% % ==================================================



% % ==================================================
% \section{Make-up exams:}
% Make-up exams will only be given with written evidence of an official university
% excused absence. 
% % Section 7.3 of the University Student Rules states that for an absence: 
% % ``to be excused the student must notify his or her instructor in writing (acknowledged email message is acceptable) prior to the date of absence if such notification is feasible. In cases where advance notification is not feasible (e.g., accident or emergency) the student must provide notification by the end of the second working day after the absence. This notification should include an explanation of why notice could not be sent prior to the class.'' 
% % ==================================================

% ==================================================
\section{Incompletes:}
A grade of ``incomplete'' may be considered if all but a small portion of the class has been successfully completed , but the student in question is prevented from completing the course by a severe, unexpected, and documented event. Students who are simply behind in their work should consider dropping the course.
% ==================================================

\pagebreak

% ==================================================
\section{Programming:}   This course contains a gentle introduction to scientific computing with Matlab and FEniCS via Python.  Matlab and Python are two of the most widely-used programming languages in science, mathematics, and engineering, and can be a very strong asset to future scientific work.  \textbf{No previous programming experience is assumed.}  Students are assumed to be able to have basic computer skills, such as using a mouse, keyboard, etc., and be able to download and install programs and navigate websites.  Basic programming in these languages will be taught in class on certain days.  Programming assignments and/or projects will be announced in class.
% ==================================================


% \pagebreak

% ==================================================
\section{ADA  Statement:} Students with disabilities are encouraged to contact the instructor for a confidential discussion of their individual needs for academic accommodation. It is the policy of the University of Nebraska-Lincoln to provide flexible and individualized accommodation to students with documented disabilities that may affect their ability to fully participate in course activities or to meet course requirements. To receive accommodation services, students must be registered with the  {\bf Services for Students with Disabilities (SSD) office}, 132 Canfield Administration, 472-3787 voice or TTY. 
% ==================================================

% % ==================================================
% \section{Copyright policy:} Printed materials disseminated in class or on the web are protected by copyright laws. One Xerox copy (or download from the web) is allowed for personal use. Multiple copies or sale of any of these materials is strictly prohibited.
% % ==================================================

% ==================================================
\section{Grade Questions:} 
Any questions regarding grading/scoring of homework, exams, or projects must be made within two class days from when they were handed back, or no change in grade will be made.  
% Any questions regarding grading/scoring of quizzes or exams must be made before the test leaves the room, or no change in grade will be made. If you need more time to look at a quiz or exam and do not want to lose your right of protest, hand it back to me at the end of class, and arrange to come to office hours.
\\\textbf{NOTE:} Because of privacy rights, \textbf{I cannot discuss grades over email or telephone. Please do not email me asking about your grade.  I will not be able to give you any information.}  Of course, I am happy to discuss grades in my office.
% ==================================================

% ==================================================
\section{Important Dates:}
\begin{tabular}{ll}
Jan. 20, 2016 (Fri): & Last day to withdraw from this course and not have\\& it appear on your transcript.\\

Mar. 3, 2016 (Fri): & Last day to change your grade option to or from Pass/No Pass.\\
    
Mar. 19-26, 2016:   & Spring break, no class.\\
Apr. 7, 2016 (Fri): &  Last day to drop this course and receive a grade of W. \\&
     (No permission required.) After this date, you cannot drop.
  \end{tabular}
% ==================================================  

% ==================================================
\section{Departmental Grading Appeals Policy:} Students who believe their academic evaluation has been prejudiced or capricious have recourse for appeals to (in order) the instructor, the departmental chair, the departmental appeals committee, and the college appeals committee.  
% ==================================================

% % ==================================================
% \section{Final Exam Policy:} Students are expected to arrange their personal and
% work schedules to allow them to take the final exam at the scheduled time.  The
% final exam for this course is:\\ 
% \textbf{Tuesday, May 3, 2016,  3:30 pm-5:30 pm (same classroom).}   
% 
% % ==================================================

% ==================================================
\section{Disclaimer:} 
While this syllabus was prepared carefully and according to information
available at the beginning of the semester, changes may be necessary in the
interest of good teaching. Changes to any of the information above will
be announced in class and posted on the class web site. This includes in
particular possible updates or corrections to the syllabus, and changes of
exam dates. Care has been taken to avoid any conflict between this syllabus and official university policy. Any such conflict, if it exists, is purely accidental, and appropriate measures will be taken to rectify any such mistake. 
% ==================================================

% % ==================================================
% \section{Exams:}
% You are required to bring and possibly present your Aggie Card or a government issued ID card when taking exams, as well as standard writing materials.  
% \begin{itemize}
% \item \textbf{Midterm Exam 1:} Tuesday June 17, Bessey Hall 108.
% \\ \underline{Material}: Chapters 1,2,8, and Sections 7.2, 7.3
% \item \textbf{Midterm Exam 2:} Tuesday July 1, Bessey Hall 108
% \\ \underline{Material}: Chapters 3, 6, and 7
% \item \textbf{Final Exam:} Monday July 7, 10:30-12:30 p.m., Bessey Hall 108
% \\ \underline{Material}: Comprehensive--all material from the course.
% \end{itemize}
% % ==================================================

\newpage 

% \newsectionwidth{1in}
% \newgeometry{textwidth=10cm,textheight=10cm}

% ==================================================
Tentative List Of Topics: The following tentive list of topics is a rough guide to the material covered in the course, but is subject to change. Updates and changes to the content will be announced in class, over email, on Canvas, or on the course website.\\

\hrule

\vspace{-0.5cm}

{\small
% \hspace*{-14.5cm}
% \begin{twocolumn}
\begin{itemize}
\item Intro to Matlab, loops in chaos
\item Euler's method for ODEs, with resolution error
\item RK-4, order of error, stability

\hrule
\item Fourier transforms and the DFT
\item FFT, Parseval
\item FFT in Matlab to compute derivatives (energy error)

\hrule

\item The heat equation and the CFL
\item Coding the heat equation
\item Burgers equation and aliasing

\hrule

\item Functional analysis review
\item Overview of finite element methods
\item Coding the stiffness matrix

\hrule

\item Solving 1D Poisson in Matlab
\item Error estimates
\item Checking error vs. resolution

\hrule

\item Lagrangian interpolation
\item Orthogonal polynomials
\item Special polynomials

\hrule

\item Newton-Cotes numerical quadrature
\item Gaussian numerical quadrature
\item 2D numerical quadrature

\hrule

\item Introduction to FEniCS
\item Weak derivatives
\item Sobolev spaces

\hrule

\item Evolution equations in FEniCS
\item Hilbert spaces
\item Boundary conditions

\hrule


\item Duality
\item Projections
\item Lax-Millgram

\hrule

\item Triangular elements
\item Transforms and the reference simplex
\item Some notes on meshes

\hrule

\item Coding Finite Element Methods
\item Data Structures
\item The unexpected difficulty of solving $Ax = b$

\hrule

\item Some comments on finite-precision arithmetic 
\item Preconditioners
\item Using software libraries

\hrule

\item The Stokes equations
\item Mixed finite elements
\item Saddle Point Problems

\hrule

\item Additional topics if time: Parallel computing basics, Finite volume methods, finite difference methods,  Symplectic integrators, IMEX methods, and more.
\end{itemize}
% \end{twocolumn}
}


\hrule

\end{resume}
\end{document} 



