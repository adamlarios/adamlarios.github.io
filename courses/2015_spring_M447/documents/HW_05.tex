% \RequirePackage[l2tabu, orthodox]{nag}
%% Checks for obsolete LaTeX packages and outdated commands. 
%% Does nothing as long as your syntax is right.
% 
% Things Ciprian mentioned:
% ------------------------
% Take data epsilon above the parabola, and see that it comes down
% 

\documentclass[12pt]{amsart}
% Document class possibilities:
%   amsart, article, book, beamer, report, letter
% Options:
%   letterpaper, a4paper,11pt,oneside, twoside, draft, twocolumn, landscape

%% For beamer class, see my beamer template.

%% ========== Options to Toggle When Compiling ==============
%\usepackage{syntonly}
%\syntaxonly
% \usepackage[notcite]{showkeys} %% Show tags and labels.
% %\usepackage{layout}            %% Show variable values controlling page layout.
%\allowdisplaybreaks[1]         %% Allow multiline displays to split.
%\nobibliography     %% Use proper citations, but do not generate bibliography.

%% ========== Select *.tex file encoding and language ==============
%\usepackage[language]{babel} %% Takes care of all language requirements.

%\usepackage[latin1]{inputenc}  %% Use with PuTTY or TeXMaker
\usepackage[utf8]{inputenc}  %% Use on most OS's, such as Ubuntu.

%% ============== Page Styles ==============
% \usepackage{fancyhdr}
% \pagestyle{fancy}
% \pagestyle{empty}

%% ============== Page Layout ==============
%% Allow extra space at the bottom of pages.
\raggedbottom     

%% Use smaller margins.
% \usepackage{fullpage}

%%Control page number placement.  \thepage is the current page number.
% \renewcommand{\headrulewidth}{0pt}
% \lhead{}
% \chead{}
% \rhead{}
% \lfoot{}
% \cfoot{\thepage}
% \rfoot{}

%\usepackage{geometry}  %% Can adjust the margins of individual pages
%% Use it like this:
%% \newgeometry{left=3cm,bottom=0.1cm}
%%     ... Lines that require margins adjusted ...
%% \restoregeometry

%% ============== Math Packages ==============
\usepackage{amsmath}
\usepackage{amsfonts}
\usepackage{amssymb}
\usepackage{amsthm}
\usepackage{mathtools} % An improvement of amsmath
\usepackage{latexsym}

%% ============ Typesetting add-ons ============
%\usepackage{siunitx} %Support for SI units, \num, \SI, etc.

%% ============== Single-Use Packages ==============
\usepackage{enumerate}
\usepackage{cancel}
\usepackage{cases}
\usepackage{empheq}
\usepackage{multicol}

%% ============== Graphics Packages ==============
%\usepackage{graphicx} %% Conflicts with pdflatex.
%\usepackage{graphics} %% Conflicts with eps files.
%\usepackage{epsfig} Allows eps files (?)

\usepackage{wrapfig}

%% Note: For using .eps graphics, use the graphicx package,
%% and in the document use, for example:
%% \begin{figure}
%%  \includegraphics[scale=0.5]{my_picture.eps}
%% \end{figure}

%% Prevent figures from appearing on a page by themselves:
%\renewcommand{\topfraction}{0.85}
%\renewcommand{\textfraction}{0.1}
%\renewcommand{\floatpagefraction}{0.75}

%% Force floats to always appear after their definition: 
%\usepackage{flafter}

%% ============== tikZ and PGF packages ==============
%\usepackage{ltxtable,tabularx,tabulary}
 
\usepackage{tikz}
\usepackage{pgf}
% \usepackage{pgfplots} %% Requires pgf 2.0 or later.
% % \usetikzlibrary{arrows, automata, backgrounds, calendar, 
% % chains,matrix, mindmap, patterns, petri, shadows, 
% % shapes.geometric,shapes.misc,
% % spy, trees}
% \pgfplotsset{compat=1.9} % Fixes some backwards compatibility warnings
\usetikzlibrary{arrows}

%% ============== Colors ==============
%% Warning: These are often a source of conflicts during compilation.
\usepackage{color}
\newcommand{\blue}[1]{{\color{blue} #1}}
\newcommand{\red}[1]{{\color{red} #1}}

%% ============== Notes ==============
\usepackage[backgroundcolor=gray!30,linecolor=black]{todonotes}
% \usepackage[disable]{todonotes}
%\listoftodos, \todo[noline]{}, \todo[inline]{}, \todo{}, \missingfigure{}
% \todo[fancyline]{}
   
%% ============== Fonts ==============
%\usepackage{bbm}  %% Non-Vanilla: Not include in many LaTeX distributions.
\usepackage{mathrsfs}
\usepackage{fontenc} %T1 font encoding
\usepackage{inputenc} %UTF-8 support
%\usepackage{babel} %Language specific commands, shortcuts, hyphenation.

\usepackage{verbatim}

%% Microtype improves spacing.  Load after fonts.
% \usepackage{microtype}

%% ============== Theorem Styles ==============
%% Note: newtheorem* prevents numbering.

\theoremstyle{plain}
\newtheorem{theorem}{Theorem}[section]
\newtheorem{proposition}[theorem]{Proposition}
\newtheorem{lemma}[theorem]{Lemma}
\newtheorem{corollary}[theorem]{Corollary}
\newtheorem*{claim}{Claim}

\theoremstyle{definition}
\newtheorem{definition}[theorem]{Definition}
\newtheorem{example}[theorem]{Example}
\newtheorem{exercise}[theorem]{Exercise}
\newtheorem{axiom}[theorem]{Axiom}

\theoremstyle{remark}
\newtheorem{remark}[theorem]{Remark}

%% ============== References ==============
\setcounter{secnumdepth}{3} %% Used to label subsections
\numberwithin{equation}{section} %% Equation numbering control.
\numberwithin{figure}{section}   %% Figure numbering control.

\usepackage[square,comma,numbers,sort&compress]{natbib}
\usepackage[colorlinks=true, pdfborder={0 0 0}]{hyperref}
\hypersetup{urlcolor=blue, citecolor=red}
\usepackage{url}

%% Reference things as 'fig. 1', 'Lemma 7', etc.
% \usepackage{cleveref}

%% Create references like 'on the following page', 'on page 23'
\usepackage{varioref} 

% usepackage[refpages]{gloss} %% Glossary

%%%%%%%%%%%%%%%%%%%%% MACROS %%%%%%%%%%%%%%%%%%%%%

% ============================== Vectors ==============================
\newcommand{\vect}[1]{\mathbf{#1}}
\newcommand{\bi}{\vect{i}}
\newcommand{\bj}{\vect{j}}
\newcommand{\bk}{\vect{k}}

\newcommand{\bn}{\vect{n}}

\newcommand{\bu}{\vect{u}}
\newcommand{\bv}{\vect{v}}
\newcommand{\bw}{\vect{w}}
\newcommand{\boldm}{\vect{m}}
\newcommand{\bx}{\vect{x}}
\newcommand{\by}{\vect{y}}
\newcommand{\bz}{\vect{z}}

\newcommand{\be}{\vect{e}}
\newcommand{\bg}{\vect{g}}

\newcommand{\bbf}{\vect{f}}

% ==================== Fields ==================
\newcommand{\field}[1]{\mathbb{#1}}
\newcommand{\nN}{\field{N}}
\newcommand{\nZ}{\field{Z}}
\newcommand{\nQ}{\field{Q}}
\newcommand{\nR}{\field{R}}
\newcommand{\nC}{\field{C}}
\newcommand{\nF}{\field{F}}
\newcommand{\nK}{\field{K}}

% ======================== Script Symbols  ========================
\newcommand{\sL}{\mathscr L}
\newcommand{\sH}{\mathscr H}
\newcommand{\sG}{\mathscr G}

% ====================== Caligraphic Symbols ======================
\newcommand{\cA}{\mathcal A}
\newcommand{\cB}{\mathcal B}
\newcommand{\cC}{\mathcal C}
\newcommand{\cD}{\mathcal D}
\newcommand{\cF}{\mathcal F}
\newcommand{\cH}{\mathcal H}

\newcommand{\cK}{\mathcal K}
\newcommand{\cL}{\mathcal L}

% ======================== Fraktur Symbols  ========================
% Note: Use mathrsfs package.

\newcommand{\fm}{\mathfrak m}

% ========================== Bold Symbols ==========================
\newcommand{\bvphi}{\boldsymbol{\vphi}}
\newcommand{\bPhi}{\boldsymbol{\Phi}}

% ======================== Misc. Symbols ========================
\newcommand{\nT}{\mathbb T}
\newcommand{\vphi}{\varphi}
\newcommand{\maps}{\rightarrow}
\newcommand{\Maps}{\longrightarrow}
\newcommand{\sand}{\quad\text{and}\quad}
\newcommand{\QED}{\hfill$\blacksquare$}
\newcommand{\tac}{\textasteriskcentered}
%\newcommand{\dhr}{\m\athrel{\lhook\joinrel\relbar\kern-.8ex\joinrel\lhook\joinrel\rightarrow}}

% ========================== Operations ==========================
\newcommand{\cnj}[1]{\overline{#1}}
\newcommand{\pd}[2]{\frac{\partial #1}{\partial #2}}
\newcommand{\npd}[3]{\frac{\partial^#3 #1}{\partial #2^#3}} %\npd{f}{x}{2}
\newcommand{\abs}[1]{\left\lvert#1\right\rvert}
%\newcommand\norm[1]{\left\vert\mkern-1.7mu\left\vert#1\right\vert\mkern-1.7mu\right\vert}
%\newcommand\bnorm[1]{\bigl\vert\mkern-2mu\bigl\vert#1\bigr\vert\mkern-2mu\bigr\vert}
\newcommand{\set}[1]{\left\{#1\right\}}
\newcommand{\ip}[2]{\left<#1,#2\right>}
\newcommand{\iip}[2]{\left<\left<#1,#2\right>\right>}
\newcommand{\braket}[1]{\left<#1\right>}
\newcommand{\pnt}[1]{\left(#1\right)}
\newcommand{\pair}[2]{\left(#1,#2\right)}

%Advection operators:
\newcommand{\adv}[2]{(#1 #2)}
\newcommand{\vectadv}[2]{\;#1 \otimes#2\;}

% ============ Special Macros For This Paper ==================
\newcommand{\diff}[1]{\widetilde{#1}}
\newcommand{\bud}{\diff{\bu}}
\newcommand{\ud}{\diff{u}}
\newcommand{\xid}{\diff{\xi}}
\newcommand{\thetad}{\diff{\theta}}

\newcommand{\bue}{\vect{u}^{\epsilon}}
\newcommand{\pe}{p^{\epsilon}}
\newcommand{\thetae}{\theta^{\epsilon}}

\newcommand{\bude}{\diff{\bue}}
\newcommand{\ude}{\diff{ue}}
\newcommand{\thetade}{\diff{\thetae}}

\newcommand{\bun}{\vect{u}^{(n)}}
\newcommand{\thetan}{\theta^{(n)}}
\newcommand{\omegan}{\omega^{(n)}}

% \newcommand{\thetaSt}{\cnj{\theta}} % 'St' for 'Steady.'
% \newcommand{\uSt}{\cnj{u}}
% \newcommand{\buSt}{\cnj{\bu}}

\newcommand{\thetaSt}{\theta} % 'St' for 'Steady.'
\newcommand{\uSt}{u}
\newcommand{\buSt}{\bu}

\newcommand{\thetaPer}{\widetilde{\theta}} % 'Per' for 'Perturbation.'
\newcommand{\uPer}{\widetilde{u}}
\newcommand{\buPer}{\widetilde{\bu}}
%\renewcommand{\section}[1]{\section{\texorpdfstring{#1}}}

\newcommand{\dist}{\text{dist}}

% Put notation for initial data here, so it can be changed easily.
\newcommand{\initialData}[1]{#1_0}
\newcommand{\buInit}{\initialData{\bu}}
\newcommand{\uInit}{\initialData{u}}
\newcommand{\uStInit}{\initialData{\uSt}}
\newcommand{\uStInitOne}{\initialData{\uSt}}
\newcommand{\uStInitTwo}{\initialData{\uSt}}

\newcommand{\thetaInit}{\initialData{\theta}}

% \newcommand{\weaklim}[1]{\substack{\mathrm{wk\mbox{-}lim}\\[0.1ex]#1}}
\DeclareMathOperator*{\weaklim}{wk-lim}

% ========================== Norms ==========================
\newcommand{\norm}[1]{\left\|#1\right\|}
\newcommand{\snorm}[2]{\left\|#1\right\|_{#2}}
\newcommand{\normH}[1]{|#1|}
\newcommand{\normV}[1]{\|#1\|}
\newcommand{\normLp}[2]{\|#2\|_{L^{#1}}}
\newcommand{\normHs}[2]{\|#2\|_{H^{#1}}}
\newcommand{\normLL}[3]{\|#3\|_{L^{#1}([0,T],L^{#2})}}
\newcommand{\normLH}[3]{\|#3\|_{L^{#1}([0,T],H^{#2})}}
\newcommand{\normCL}[3]{\|#3\|_{C^{#1}([0,T],L^{#2})}}
\newcommand{\normCH}[3]{\|#3\|_{C^{#1}([0,T],H^{#2})}}
% \documentclass[final,11pt]{article}
% %\usepackage{amstex} 
% \usepackage{amsmath}
% \usepackage{epsfig} 
% \usepackage{latexsym}
\usepackage{epic}
\usepackage{eepic}
% \usepackage{graphicx}
% \usepackage{amssymb}
% \usepackage{epstopdf}
\usepackage{nicefrac}
\newcommand{\f}{\frac}
\newcommand{\eps}{\epsilon}

\usepackage{marvosym}
% ======================= \texttt{MATLAB} =======================
\usepackage{listings} % Use for code.
\usepackage{textcomp} % Used for upquote.
% \usepackage{color} 
\definecolor{dkgreen}{rgb}{0,0.6,0}
\definecolor{gray}{rgb}{0.5,0.5,0.5}

\lstset{language=Matlab,
   keywords={break,case,catch,continue,else,elseif,end,for,function,
       global,if,otherwise,persistent,return,switch,try,while},
   basicstyle=\ttfamily,
   keywordstyle=\color{blue},
   commentstyle=\color{red},
   stringstyle=\color{dkgreen},
%    numbers=left,%    numberstyle=\tiny\color{gray},
   stepnumber=1,
   numbersep=10pt,
   backgroundcolor=\color{white},
   tabsize=4,
   showspaces=false,
   showstringspaces=false,
   commentstyle=\color{dkgreen}
   }

\begin{document}
\begin{center}
{\bf MATH 447/847 - Numerical Analysis\\
Homework \#5 \\
Iterative Methods and Gershgorin Disks}
\end{center}

\noindent
Note: All the problems here, except for maybe 2(a), are very short, so look for simple solutions, rather than complicated ones!

\bigskip


% \small  
% \bigskip
% \noindent
% Solve any set of problems for 100 points.
% %The homework should be presented at the beginning of the class.
% 5 pts per day penalty for delay of the homework applies.
%  
% \bigskip
% \noindent


\begin{itemize}
\item[\bf Problem 0.] Read pages 186-188  in Trefethen and Bau (read all of Chapter 24 if you want a good review of eigenvalues, diagonalizability, and geometric/algebraic multiplicity).

\bigskip

\item[\bf Problem 1] Given a square matrix $A=(a_{ij})_{i,j=1}^n$, let us define its  \textbf{Gerschgorin disks} for $i=1,\ldots,n$ by:
 \begin{align*}
  D_i = \Bigg\{z\in\nC : |z-a_{ii}|\leq \sum_{\substack{j=1\\j\neq i}}^n|a_{ij}|\Bigg\}
 \end{align*}
 These are disks in the complex plane $\nC$, which are centered at the diagonal entries $a_{ii}$, and whose radius is the sum of the absolute values of the off-diagonal entries in the $i^{\text{th}}$ row. They are very important tools in numerical analysis. 
 
 Draw a picture (on the same plane) of the Gershgorin disks $D_1$, $D_2$, $D_3$ for the matrix
 \begin{align*}
  A =  \begin{bmatrix}
  2 & 1&1\\1 & 2&1\\1 & 2&3
 \end{bmatrix}.
 \end{align*}
% Without computing the eigenvalues of the matrix (it's not worth your time), tell whether the matrix could have an eigenvalue greater than 6, or less than zero, based on your picture.
 
 \bigskip
 
 \item[\bf Problem 2(a)] Do problem 24.2(a) and 24.2(c) in Trefethen and Bau. 
 
 \bigskip
 
  \item[\bf Problem 2(b)]
 Gershgorin's Theorem (sometimes called Gershgorin's Localization Theorem) says that all the eigenvalues of a matrix must live in the Gershgorin disks.  Without computing the eigenvalues of the matrix (it's not worth your time), tell whether the matrix in Problem 1 could have an eigenvalue greater than 6, or less than zero, based on your picture.
 
 \bigskip
 
\item[\bf Problem 3] Consider trying to solve the problem $A\bx=\mathbf{b}$ by an iteration method $\bx^{k+1} = G\bx^k+b$ for some $G$.  If we write $A=D-L-U$, where $D$ is a diagonal matrix made from the diagonal entries of $A$, $-L$ are the lower entries of $A$, and $-U$ are the upper entries of $A$ then the Jacobi method is to choose $G = D^{-1}(L+U)$.  That is, given $A=(a_{ij})_{i,j=1}^n$, set $G$ to be
\begin{align*}
 G = \begin{bmatrix}
      0 & -\frac{a_{12}}{a_{11}} & -\frac{a_{13}}{a_{11}} & \cdots & -\frac{a_{1,n}}{a_{11}} \\
      -\frac{a_{21}}{a_{22}} & 0 & -\frac{a_{23}}{a_{22}} & \cdots & -\frac{a_{2,n}}{a_{22}} \\
      -\frac{a_{31}}{a_{33}} & -\frac{a_{32}}{a_{33}} & 0 & \cdots & -\frac{a_{3,n}}{a_{33}} \\
      \vdots &\vdots& \vdots & \ddots & \vdots \\
      -\frac{a_{n1}}{a_{nn}} & -\frac{a_{n2}}{a_{nn}} & -\frac{a_{n3}}{a_{nn}} & \cdots & 0
     \end{bmatrix}.
\end{align*}

% \begin{align*}
%  A = \begin{bmatrix}
%       a_{11} & a_{12} & a_{13} & \cdots & a_{1,n} \\
%       a_{21} & a_{22} & a_{23} & \cdots & a_{2,n} \\
%       a_{31} & a_{32} & a_{33} & \cdots & a_{3,n} \\
%       \vdots &\vdots& \vdots & \ddots & \vdots \\
%       a_{n1} & a_{n2} & a_{n3} & \cdots & a_{n,n}
%      \end{bmatrix}
% \end{align*}

Recall that a strictly-diagonally-dominant (SDD) matrix $A=(a_{ij})_{i,j=1}^n$, is a matrix such that
 \begin{align*}
  |a_{ii}|> \sum_{\substack{j=1\\j\neq i}}^n|a_{ij}| \text{ for all }j=1,\ldots,n.
 \end{align*}

 Prove that if $A$ is SDD, then the Jacobi method converges.  [Hint: Use the Fundamental Theorem of Iterative Methods and Gershgorin's Theorem, which together make the proof just a couple of lines.]
 
\bigskip
 
\item[\bf Problem 4(a)] Consider Richard's iteration method with the scaling parameter $\tau>0$ ($A^{-1}\approx B=\tau I$, $Q=B^{-1}=\frac{1}{\tau}I$), so that the iteration $\bx^{k+1} = (I-BA)\bx^k+Bb$ is just
\begin{align*}
 \bx^{k+1} = (I-\tau A) \bx^k+\tau b,
\end{align*}
so that $G = (I-\tau A)$.  Recall the error equation $\be^{k+1} = G\be^k$.  Show that 
\begin{align*}
 \|\be^{k+1}\|_2\leq \pnt{\max_{i}|1-\tau\lambda_i|}\|\be^{k}\|_2
\end{align*}
where $\lambda_i$ are the eigenvalues of $A$.  

\bigskip

\item[\bf Problem 4(b)] Let us order the eigenvalues so that $\lambda_1\leq\lambda_2\leq\cdots\leq\lambda_n$.  Show that
\begin{align*}
 \max_{i}|1-\tau\lambda_i| = \max\set{|1-\tau\lambda_1|,|1-\tau\lambda_n|}
\end{align*}
and that this quantity is \textbf{smallest} when $1-\tau\lambda_1 = -1+\tau\lambda_n$.  (Hint: Don't use calculus to find the min of the max, it is ugly.  Instead, just draw pictures of $|1-\tau\lambda_i|$.) Also, solve for $\tau$ in this case.



\end{itemize}



 
 
 
\end{document}
 